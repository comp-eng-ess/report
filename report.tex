\documentclass[17pt]{extarticle}
\usepackage[a4paper, total={6in, 8in}]{geometry}
\usepackage{fontspec}
\usepackage{babel}
\babelprovide[main, import]{thai}
\babelfont{rm}{TH Sarabun New}
\usepackage{hyperref}
% \setmainfont{TH Sarabun New}
\setmainfont{THSarabunNew.ttf}
\setmonofont{THSarabunNew.ttf}
\geometry{margin=1in}

\title{CP Rate My Professor}
\author{Naphat Darunaitorn, Noppakorn Jiravaranun,\\ Nopparuj Poonsubanan, Ittipat Yodprasit}
\begin{document}
\maketitle
\section{ที่มาและวัตถุประสงค์}

จากสถานการณ์โรคระบาดในปัจจุบัน
ก่อให้เกิดปัญหาด้านการศึกษาในหลายๆด้านทั้งจากส่วนบุคคลากร และ ระบบการศึกษาแบบออนไลน์
ทั้งนี้ทางคณะผู้จัดทำได้เห็นปัญหานี้และอยากสะท้อนความเห็นจากผู้เรียนต่อรูปแบบการการสอน
เกณฑ์การให้คะแนน และส่วนอื่นๆ ของผู้สอนแต่ละท่าน ผ่านหน้าเว็บของเรา
โดยมีการออกแบบให้ง่ายต่อการแสดงความคิดเห็น และสามารถแสดงข้อมูลแบบเข้าใจง่าย
เพื่อให้ผู้สอนและบุคลากรที่เกี่ยวข้องสามารถนำข้อมูลไปวิเคราะห์เพื่อนำไปปรับปรุงพัฒนา
เพื่อระบบการศึกษาในภาวะวิกฤตินี้ให้ดียิ่งขึ้น

\section{การใช้งานเว็บ}

หน้าเว็บ CP Rate My Professor จะเป็นเว็บสำหรับให้คะแนนและแสดงความคิดเห็นต่อการเรียนการสอนของอาจารย์แต่ละท่าน
โดยในหน้าหลักจะเป็นรายชื่อของอาจารย์ทั้งหมด มีช่องค้นหาเพื่อความง่ายต่อการค้นหาชื่ออาจารย์
สามารถคลิกที่ชื่ออาจารย์แต่ละท่าน เพื่อเข้าไปตรวจสอบ และแสดงความคิดเห็นได้
โดยในความคิดเห็นหนึ่งๆจะประกอบด้วยคะแนนจาก0ถึง5 รหัสวิชาที่เรียนกับอาจารย์ท่านนั้นๆ ตอนเรียน ภาคเรียน ปีการศึกษา และ ความคิดเห็นของผู้เรียน
โดยมีคะแนนเฉลี่ยในภาครวม และ จำนวนความคิดเห็นรวม แสดงอยู่ในหน้านี้ด้วย สามารถกดย้อนกลับไปหน้าหลักเพื่อดูหรือแสดงความคิดเห็นต่ออาจารย์ท่านอื่นๆได้

\section{เทคโนโลยีที่ใช้ในเว็บ}
\url{https://cp-rate-my-professor.web.app/}
\subsection{หน้าหลัก}

มีการใช้ navigation bar ที่ส่วนบนของหน้าเว็บ มี search box ที่จะกรองชื่ออาจารย์ทุกครั้งที่ตัวอักษรในช่องค้นหาถูกเปลี่ยนแปลง มีส่วนกล่องชื่ออาจารย์ถูกดึงมาจาก database ของ firebase
ที่สามารถคลิกเพื่อเข้าสู่หน้ารวมคะแนนและความคิดเห็นของอาจารย์แต่ละท่าน

\subsection{หน้ารวมคะแนนและความคิดเห็นของอาจารย์แต่ละท่าน}

ส่วนบนยังมี navigation bar เหมือนหน้าหลัก มีการแสดงชื่ออาจารย์ท่านนั้นๆ มีส่วนการแสดงคะแนนเฉลี่ยและจำนวนความคิดเห็น ปุ่มกดเพื่อเพิ่มความคิดเห็น เพื่อเข้าสู้หน้าแสดงความคิดเห็น ในส่วนล่างมีกล่องความคิดเห็นเพื่อแสดงความคิดเห็นจากผู้เรียนที่เขียนไว้ก่อนหน้าถูกดึงมาจาก database ของ firebase

\subsection{หน้าแสดงความคิดเห็น}

ส่วนบนมีการแสดงชื่ออาจารย์ที่กำลังถูกแสดงความคิดเห็น มีช่องสำหรับกรอกคะแนน รหัสวิชา ตอนเรียน ภาคเรียน ปีการศึกษา และช่องแสดงความคิดเห็นและมีปุ่ม submit เพื่อส่งข้อมูลที่กรอกไปยัง database เพื่อแสดงผลต่อไป
ในส่วนของหน้าเว็บเพจในการเข้าไปในหน้าอื่นนอกเหนือจากหน้าหลักจะมีการวาดหน้าเว็บใหม่บนหน้าเดิม เพื่อให้ถูกต้องตามวัตถุประสงค์ single-page web application


\pagebreak
\section{Basic requirement}
ทางกลุ่มเราตกลงกันทำเว็บประเมินอาจารย์เพื่อสะท้อนความคิดเห็นของผู้เรียนต่ออาจารย์ โดยคาดหวังให้ผู้เรียนสะท้อนข้อดีและข้อเสียของการสอนในรูปแบบต่างๆของอาจารย์แต่ละท่านเพื่อทั้งตัวอาจารย์และบุคลากรที่เกี่ยวข้องสามารถนำข้อมูลไปปรับใช้หรือพัฒนาการสอน ผู้เรียนรุ่นต่อไปสามารถพิจารณาเข้าเรียนจากลักษณะการสอนของอาจารย์แต่ละท่านที่ตรงกับวิธีการเรียนของแต่ละคนได้ เราทำ single-page web application ให้มีหลายหน้า (home, about, contact ,หน้ารายละเอียดของอาจารย์แต่ละท่าน) โดยการวาด elements ในหน้าเดิม โดยทั้งส่วนหน้าบ้านและหลังบ้านใช้ libraries/frameworks/plugins เพียงในส่วนของ activity 6 และ 7 เท่านั้น โดยสมาชิกกลุ่มได้มีการวางแผนและแบ่งงาน มีการนัดประชุมเพื่ออัพเดทงานที่เหลือและสรุปความคืบหน้าตลอดช่วงการทำงาน
\section{Challenging requirements}

การออกแบบเว็บไซต์สำหรับแสดงผลในหน้าจอหลายขนาด เริ่มทำโดยออกแบบเว็บไซต์ที่ใช้บนpcก่อน แล้วจึงไปปรับใช้ในการออกแบบของมือถือผ่านคำสั่ง@media (max-width: 480px)
จากนั้นปรับ style ใน css ให้เหมาะกับมือถือในแต่ละรุ่นในขั้นตอนสุดท้าย

\section{ขั้นตอนการทำงาน}

ในส่วนของการออกแบบหน้าเว็บของเราพยายามให้ Interface มีความเข้าถึงผู้ใช้ได้ดี และเข้าใจง่าย โดยเรามีการ Design ผ่าน Figma ก่อนมาทำใน css

\pagebreak
\section{หน้าที่รับผิดชอบ}
\begin{enumerate}
    \item Naphat Darunaitorn
          \begin{itemize}
              \item Project Manager
              \item Organizer
              \item Website Feature Design
              \item Report
          \end{itemize}
    \item Noppakorn Jiravaranun - Frontend Development, Backend Design and Development
    \item Nopparuj Poonsubanan - Frontend Development, Backend Development
    \item Ittipat Yodprasit - Frontend Design and Development
\end{enumerate}

\end{document}